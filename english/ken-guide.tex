% Copyright (c) 2013 Datalogisk Kantineforening.
% Licenseret under EUPL, version 1.1 udelukkende.
% Licensteksten: https://joinup.ec.europa.eu/software/page/eupl/licence-eupl

\documentclass{article}

\usepackage[english]{skiltefabrikken}

\begin{document}

\hovedoverskrift{KEN Guide}

\maketitle

\fontsize{18}{18}\selectfont

\begin{enumerate}

\setcounter{enumi}{-1}

\bfseries \item Rinse the dirty items thoroughly \normalfont
\begin{itemize}

\item Please ensure that no coffee rings or bits of food remain. KEN is a
disinfecting machine and not a dishwasher.

\end{itemize}

\bfseries \item Place the rinsed items in the appropriate tray by the sink.
\normalfont

\begin{itemize}

\item Cups, bowls and other concave objects must be placed facing downward,
such that they do not get filled up with water.

\item Cutlery should be placed in a cutlery container.

\item Do not place items on top of one another -- they won't get cleaned!

\end{itemize}

\bfseries \item If the tray is full: \normalfont

\begin{enumerate}

\item Remove the clean tray from KEN and place it on the clean side, i.e. by
the plates

\item Put the full tray in KEN. Start KEN by pressing the button on the front.

\item Find a new tray. These are located above the cup trolleys.

\end{enumerate}

\bfseries \item If there are already trays on the clean side: \normalfont

\begin{itemize}

\item Put the clean items and trays away.

\end{itemize}

\bfseries \item If you are the last person to leave the canteen: \normalfont

\begin{itemize}

\item Turn KEN off and carefully pull out the cylindrical plug inside KEN to
drain the dirty water.

\end{itemize}

\end{enumerate}

\underskriv

\end{document}

