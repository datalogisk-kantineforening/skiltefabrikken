\documentclass{article}

\usepackage[english]{skiltefabrikken}

\begin{document}

\hovedoverskrift{Coffee Guide}

\maketitle

\null
\vspace{-0.2cm}

\fontsize{18}{20}\selectfont

\vspace{-1.7cm}

\hspace{-0.9cm}\noindent How to brew DIKU-coffee

\vspace{1.5cm}

\begin{enumerate}

\setcounter{enumi}{-1}

\itemsep-0.1cm

\item Check that there's no coffee left first!

\item Remove the filter unit from the machine, and replace the used coffee
filter with a fresh one.

\item Put the appropriate amount of coffee grounds in the filter:

\begin{itemize}

\bfseries

\item For 1 litre use 1-2 level decilitre measures

\item For 3 litres use 3 level decilitre measures

\item For 5 litres use 5-6 level decilitre measures

\end{itemize}

\item Put the filter unit back on the machine, and line the water pipe up with
the hole on the top of the filter unit.

\item Press the button corresponding to the amount of coffee you're brewing.

\item Hang the sign indicating that the coffee is brewing on the grey tap.

\item Set the current time on the circular ``last brewed'' indicator.

\item The coffee's done when the machine beeps thrice.

\item Grab a mug and enjoy a delicious cup of DIKU coffee.

\end{enumerate}

\vspace{0.2cm}

\begin{center}
\LARGE

No filters/coffee grounds? Try looking in the cupboard above the cup trolleys.
If you can't find what you need there, then find a member of the board, or send
an email to \texttt{bestyrelsen@kantinen.org}.

\end{center}

\underskriv

\end{document}

