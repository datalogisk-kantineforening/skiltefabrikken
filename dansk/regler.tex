% Copyright (c) 2013 Datalogisk Kantineforening.
% Licenseret under EUPL, version 1.1 udelukkende.
% Licensteksten: https://joinup.ec.europa.eu/software/page/eupl/licence-eupl

\documentclass{article}

\usepackage[english]{skiltefabrikken}

\begin{document}

\hovedoverskrift{Kantineregler}

\maketitle

\null
\vspace{-0.2cm}

{\fontfamily{cmr}\fontsize{13}{15}\selectfont

\begin{enumerate}

\item Efterladt mad skal markeres med navn og dato. Umarkeret mad og mad der er
blevet dårligt smides ud. Kantinen har to brugerkøleskabe samt en fryser der
kan benyttes til opbevaring. Buffetkøleskabet må ikke anvendes til opbevaring.

\item Der betales ikke pant for flasker der er købt i kantinen, så tomme
flasker og dåser skal afleveres tilbage og sorteres korrekt i de pantbeholdere
der er placeret i nordenden.

\item Alle er velkomne til at bruge køkkenet til madlavning hvis de rydder
ordenligt op efter sig. Ved større madlavningsprojekter bedes man aftale det
med bestyrelsen på forhånd.

\item Vis hensyn over for andre folk i kantinen. Undgå derfor unødig støj i
instituttets åbningstider. Brug gerne høretelefoner.

\item Ryd op efter dig selv og fjern dit affald. Papaffald der ikke har været
mad i kan smides det tilegnede bur i sydenden. Glasaffald skal i skraldespanden
under vasken i køkkenet.

\item Stil ting tilbage og vask dem op. Kopper, tallerkener o.l. skal ikke
efterlades andre steder på instituttet.

\item Er bakken med opvask fyldt? Start Ken, og tøm ham hvis han er fyldt.

\item Lav en ny omgang kaffe hvis du tager det sidste. Undgå dog at lave 5 L
hvis du sidder ene person om aftenen.

\item Komfurene og ovnene skal altid stå frie hvis ikke de benyttes. Hvis I har
noget mad der skal køle ned, så placer det venligst på bordet foran komfurene
(tæt på Amanda), og smid evt. et skilt med jeres navn på.

\item Af hensyn til brandsikkerhed må kantinen ikke forlades af madlaverne
imens de tilbereder mad.

\item At tage mad fra buffetten, kantinens køleskabe, automaterne m.m. uden at
betale for det, betragtes som tyveri.

\end{enumerate}

\underskriv

\end{document}

