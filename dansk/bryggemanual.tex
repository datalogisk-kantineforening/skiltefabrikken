\documentclass{article}

\usepackage{skiltefabrikken}

\begin{document}

\hovedoverskrift{Bryggemanual}

\maketitle

\null
\vspace{-0.2cm}

\fontsize{18}{20}\selectfont

\vspace{-1.7cm}
\hspace{-0.9cm}\noindent Vejledning i brygning af DIKU-kaffe

\vspace{1.5cm}

\begin{enumerate}

\setcounter{enumi}{-1}

\itemsep0em

\item Se om bryggesøjlsen er tom, hvis ikke så burde du ikke lave kaffe.

\item Tag filterbeholderen af bryggesøjlen, smid det brugte filter ud og sæt et
nyt i.

\item Tag kaffedåsen og fyld kaffe i filteret:

\begin{itemize}

\bfseries

\item Til 1 liter bruges 1-2 strøgne dl-mål

\item Til 3 liter bruges 3 strøgne dl-mål

\item Til 5 liter bruges 5-6 strøgne dl-mål


\end{itemize} \normalfont

\item Placer filterbeholderen på bryggesøjlen og placer vandrøret lige over
hullet på filterbeholderen.

\item Tryk på knappen hørende til den mængde kaffe du er ved at brygge.

\item Sæt bryggeskilten på kaffehanen.

\item Sæt p-skiven.

\item Når du hører tre bip, er kaffen klar.

\item Tag en kop og nyde den himmelske aromatiske væske.

\end{enumerate}

\vspace{0.2cm}

\LARGE

\begin{center}

Mangler der filtre eller kaffe, så kig i skabet over kopperne.

I krisesituationer find en bestyrelsesmedlem eller send en email til
\texttt{kantine@diku.dk}.

\end{center}

\underskriv

\end{document}
